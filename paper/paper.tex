\documentclass[sigconf]{acmart}
\AtBeginDocument{%
  \providecommand\BibTeX{{%
    \normalfont B\kern-0.5em{\scshape i\kern-0.25em b}\kern-0.8em\TeX}}} %chktex 1

\usepackage{etoolbox} 
\apptocmd{\sloppy}{\hbadness 10000\relax} {}{} %chktex 1

\usepackage[skip=6pt plus1pt]{parskip} %% add space between the paragraphs

\usepackage{graphicx} %% For pictures
\usepackage[section]{placeins} %% used to keep the images in the section that I want them in

\begin{document}

\title{Music Versus No Music Effectiveness On Cognitive Response Time and Typing Efficiency}

\author{Jonquill Howlett}
\affiliation{%
  \institution{Colorado State University}
  \city{Berthoud}
  \state{Colorado}
  \country{USA}
  \postcode{80513-0650} %chktex 8
}
\email{jonquill.howlett@colostate.edu}

\author{Lianna Hoag}
\affiliation{%
  \institution{Colorado State University}
  \city{Fort Collins}
  \state{Colorado}
  \country{USA}
  \postcode{80525-0650} %chktex 8 %the rest of the zip will need to be corrected
}
\email{jonquill.howlett@colostate.edu} %update email 

\author{Shea Spalding}
\affiliation{%
  \institution{Colorado State University}
  \city{Fort Collins}
  \state{Colorado}
  \country{USA}
  \postcode{80525-0650} %chktex 8 %the rest of the zip will need to be corrected
}
\email{jonquill.howlett@colostate.edu} %update email 

\renewcommand{\shortauthors}{J. Howlett} %I don't know how to format this for three authors

\begin{abstract}
abstract
\end{abstract}

\ccsdesc{Music, Typing Efficiency} % Update this
\keywords{Typing Efficiency with Music, Cognitive Function} % placeholder text for the time being
\maketitle

\section{Introduction}
This project will focus on people's ability to type with distractions such as different types of music playing. Then it will test their cognitive ability to focus on a hand/eye coordination test in which we will test how well they can match shapes by measuring speed and accuracy. 
The two-part test aims to help test if the music is too distracting or will help the people be more efficient at their work. 
This is important because many people do their homework or jobs while listening to music and should they make a mistake, it could end up costing someone their grade or the mistake at work has a cascading effect to cause more errors in their work. 
The two-part test will help determine if there is a correlation between the effectiveness of music versus no music on cognitive response time and typing efficiency.

\section{Methodology}
After obtaining consent from all participants, they will proceed to take the typing test. The test will present lyrics from popular songs within a graphical user interface (GUI). The interface will automatically calculate typing speed and accuracy in the program's backend, allowing participants to focus solely on completing the test. Each participant will complete the typing test under three conditions: Music with lyrics, Music without lyrics, and No music (silent condition).
Multiple levels of music intensity may be utilized within each condition to assess varying degrees of distraction.

Following the typing test, participants will proceed to the cognitive task component of the study, which will also be conducted under the three aforementioned conditions. % Further details on the cognitive task will be provided in subsequent sections. They will then move on the the QBTest?. % This will need to be re-written.

This is a placeholder for a citation cos otherwise the compiler throws errors \cite{AudioDistractionsAshley}.

\subsection[short]{Participants}
Participants

\subsection[short]{Apparatus}
Apparatus

\subsection[short]{Procedure}
Procedure

\subsection[short]{Design}
Design

\section{Results and Discussion}
Results and Discussion

\section{Conclusion}
Conclusion


\bibliographystyle{ACM-Reference-Format}
\bibliography{paper}

\end{document}
