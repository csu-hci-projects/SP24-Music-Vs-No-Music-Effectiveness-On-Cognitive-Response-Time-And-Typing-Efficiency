\documentclass[manuscript, screen, review]{acmart} %removing review removes the numbers on left side
% documentclass was [sigconf]{acmart} changed to match with the given overleaf link

\AtBeginDocument{%
  \providecommand\BibTeX{{%  
    \normalfont B\kern-0.5em{\scshape i\kern-0.25em b}\kern-0.8em\TeX}}} %chktex 1

\usepackage{etoolbox} 
\apptocmd{\sloppy}{\hbadness 10000\relax} {}{} %chktex 1

\usepackage[skip=6pt plus1pt]{parskip} %% add space between the paragraphs

\usepackage{graphicx} %% For pictures
\usepackage[section]{placeins} %% used to keep the images in the section that I want them in

\setcopyright{acmcopyright}
\copyrightyear{2024}
\acmYear{2024}
\acmDOI{XXXXXXX.XXXXXXX}

\begin{document}

\title{Music Versus No Music Effectiveness On Cognitive Response Time and Typing Efficiency}

\author{Jonquill Howlett}
\email{jonquill.howlett@colostate.edu}
\affiliation{%
  \institution{Colorado State University}
  \city{Berthoud}
  \state{Colorado}
  \country{USA}
  \postcode{80513-0650} %chktex 8
}

\author{Lianna Hoag}
\email{jonquill.howlett@colostate.edu} %update email 
\affiliation{%
  \institution{Colorado State University}
  \city{Fort Collins}
  \state{Colorado}
  \country{USA}
  \postcode{80525-0650} %chktex 8 %the rest of the zip will need to be corrected
}

\author{Shea Spalding}
\email{jonquill.howlett@colostate.edu} %update email 
\affiliation{%
  \institution{Colorado State University}
  \city{Fort Collins}
  \state{Colorado}
  \country{USA}
  \postcode{80525-0650} %chktex 8 %the rest of the zip will need to be corrected
}

\renewcommand{\shortauthors}{Howlett, Hoag, Spalding} %I don't know how to format this for three authors 
% the template does not show any shortauthors

\begin{abstract}
The abstract does not get added until after the project is finished or nearly finished.
\end{abstract}

\ccsdesc{Music, Typing Efficiency} % Update this
\keywords{Typing Efficiency with Music, Cognitive Function} % placeholder text for the time being
\maketitle

\section{Introduction}
This project will focus on people's ability to type with distractions such as different types of music playing. Then it will test their cognitive ability to focus on a hand/eye coordination test in which we will test how well they can match shapes by measuring speed and accuracy. 
The two-part test aims to help test if the music is too distracting or will help the people be more efficient at their work. 
This is important because many people do their homework or jobs while listening to music and should they make a mistake, it could end up costing someone their grade or the mistake at work has a cascading effect to cause more errors in their work. 
The two-part test will help determine if there is a correlation between the effectiveness of music versus no music on cognitive response time and typing efficiency.

\section{Methodology}
After obtaining consent from all participants, they will proceed to take the typing test. The test will present lyrics from popular songs within a graphical user interface (GUI). The interface will automatically calculate typing speed and accuracy in the program's backend, allowing participants to focus solely on completing the test. Each participant will complete the typing test under three conditions: Music with lyrics, Music without lyrics, and No music (silent condition).
Multiple levels of music intensity may be utilized within each condition to assess varying degrees of distraction.

Following the typing test, participants will proceed to the cognitive task component of the study, which will also be conducted under the three aforementioned conditions. % Further details on the cognitive task will be provided in subsequent sections. They will then move on the the QBTest?. % This will need to be re-written.

This is a placeholder for a citation cos otherwise the compiler throws errors~\cite{AudioDistractionsAshley}.

\section{Eye Tracking Introduction}
Eye tracking technology, such as the EyeWriter 2.0, has become increasingly valuable in cognitive research, allowing for precise measurement and analysis of visual attention and eye movements. In this study, we incorporate the EyeWriter 2.0 to track participants' eye movements during the QBTest, a hand-eye coordination task.

The EyeWriter 2.0 offers high-resolution tracking, capturing gaze patterns and fixations with accuracy. By analyzing participants' eye movements during the QBTest under different music conditions, we aim to understand how distractions, such as music, influence visual attention and cognitive performance.

Understanding the impact of music on cognitive response time and typing efficiency, as measured by the QBTest and eye tracking, can provide valuable insights for improving work and study environments where music is commonly used as a background stimulus.

\section{QBTest Introduction}
The Quantified Behavioral Test (QBTest) is a well-established tool for assessing cognitive function, particularly in measuring hand-eye coordination, speed, and accuracy. Developed as a standardized measure, the QBTest provides valuable insights into cognitive performance across different tasks and conditions.

In this study, the QBTest serves as a central component for evaluating participants' cognitive response time and typing efficiency under various music conditions. By administering the QBTest alongside eye tracking using EyeWriter 2.0, we aim to understand the impact of music on cognitive function and task performance.

The QBTest's ability to quantify cognitive performance makes it an ideal tool for this study, allowing us to measure the effects of music on participants' ability to focus, process information, and perform tasks accurately. Through this research, we seek to provide insights that can enhance our understanding of how environmental factors, such as music, influence cognitive processes.

\section{Typing Test Introduction}
Typing proficiency is a fundamental skill in today's digital age, essential for academic and professional success. However, various factors can influence typing efficiency, including environmental stimuli such as music. Understanding how music affects typing speed and accuracy is crucial for optimizing work and study environments.

This study uses a typing test to assess participants' typing performance under different music conditions. The test presents participants with lyrics from popular songs, simulating a real-world typing scenario where individuals may listen to music while working or studying.

By measuring typing speed and accuracy under conditions of music with lyrics, music without lyrics, and no music (silent condition), we aim to determine the impact of music on typing efficiency. This research will contribute to our understanding of how environmental factors, such as music, can affect cognitive processes related to typing and may have practical implications for improving productivity in various settings.


\subsection[short]{Participants}
Participants

\subsection[short]{Apparatus}
Apparatus

\subsection[short]{Procedure}
Procedure

\subsection[short]{Design}
Design

\section{Results and Discussion}
Results and Discussion

\section{Conclusion}
Conclusion


\bibliographystyle{ACM-Reference-Format}
\bibliography{paper}

\end{document}
